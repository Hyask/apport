\documentclass[DIV12,halfparskip]{scrartcl}
\usepackage{booktabs}

\title{The Apport crash report format\\
  \vspace{1ex}\large Version 0.2}
\author{Martin Pitt \texttt{<martin.pitt@ubuntu.com>}}

\begin{document}
\maketitle

\tableofcontents

\newpage
\section{Introduction}

Apport is a system for automatic problem reporting and feedback, with the
following features:

\begin{itemize}
    \item intercept program crashes right when they happen the first time
    \item collect potentially useful information about the crash and the OS environment
    \item can be automatically invoked for unhandled exceptions in other programming languages (e. g. for Python)
    \item can be automatically invoked for other problems that can be
        detected mechanically (such as package installation/upgrade failures from update-manager)
    \item easy to understand UI that informs the user about the crash and instructs them on how to proceed,
    \item written in a very modular way: user interfaces (such as Gtk/Qt),
    crash databases (such as Launchpad/Bugzilla), packaging systems
    (apt/dpkg/rpm), are all factorized
    \item independent of a particular desktop environment, Linux flavour, etc.
    \item very robust due to exhaustive test suite coverage
    \item includes tools for post-processing crashes, such as post-mortem
    generation of symbolic stack traces, tools to create and work in chroots
    with only user privileges (using \verb!fakeroot! and \verb!fakechroot!)
\end{itemize}

The Apport home page\footnote{https://wiki.ubuntu.com/Apport} has some more
information.

All components of apport (crash interception, enrichment with information, UI
presentation, crash database up/download, crash post-processing) work on a
common report file format. This allows adopters of Apport to use only some
parts and combine it with existing project specific solutions like as Gnome's
bug-buddy, and get the option to eventually merge such systems.

This document describes the structure of the report files and the pre-defined data
fields.

\section{File format}

\subsection{Structure}

Apport report files consist of key/value pairs encoded with the standard
RFC822\footnote{http://www.ietf.org/rfc/rfc0822.txt} header format. Key name
and value are separated by a dot and a space (``\verb!: !'' ).

There must not be any blank lines and no lines that start with a non-whitespace
character and do not start with a key name and a colon.

\subsection{Keys}

Key names consist of numbers (\verb!0! -- \verb!9!), English letters (\verb!a!
-- \verb!z! and \verb!A! -- \verb!Z!) and dots (\verb!.!).

\subsection{Textual values}

Single line textual values directly follow the key name, colon and dot without
any further encoding or escaping. There is no line length limit.

In multi-line textual values, the line feed character (\verb!\n!, ASCII Code
10) is escaped by appending a single space (ASCII code 32). In other words,
every line of a multi-line value but the first one must be indented by a single
space which is not part of the value.

\subsection{Binary values}

This is a compressed format intended for binary data such as memory dumps. It
can optionally be used for long textual values like large log files if they
should be compressed.

A binary value is introduced by the text ``\verb!base64!'' and a line break
following the key name, colon, and space. After that, the binary data is
encoded as follows:

\begin{itemize}
    \item Write a gzip header
    \item Initialize a zlib compressor object.
    \item Read a block of (at most) 1 MiB (1,048,576 Bytes) of binary data.
    \item Compress this block with the zlib compressor.
    \item Generate the base64-encoding of the compressed block
    \item Write a space and the base64-encoded block to the report file.
    \item If there is more source data to be encoded, go to 2.
    \item flush the zlib compressor, append the gzip trailer, base64-encode the
    tail and write it to the report file, again with a space prefix.
\end{itemize}

With this algorithm the binary encoding format obeys the same text line folding
convention than the textual values.

\subsection{Ordering}

In order to keep the report files readable by humans, the following conventions
should be met:

\begin{itemize}
    \item The textual values should be at the top, the binary values at the
    bottom of the file. This eases their inspection in web browsers, even with
    partial downloads.
    \item Within each group (textual/binary), the keys should appear in
    ascending alphabetical order.
\end{itemize}

Software that processes Apport crash report files must not rely on those
conventions. It is acceptable to not follow them for performance reasons.

\subsection{Example}

This table shows an example data set:

\begin{tabular}{lp{10cm}}\toprule
\textbf{Key} & \textbf{Value}\\
\midrule
Short1 & \verb!Single line value!\\
Date & \verb!December 24, 2000!\\
Long & \verb!Multiple lines!\par\verb! with leading!\par\verb!space!\\
TestBin & \verb!ABABABABABABABABABAB\0\0\0\0\0\0\0\0\0\0ZZZZZZZZZZ!\\
\bottomrule
\end{tabular}

This would be encoded as:

\begin{verbatim}
Date: December 24, 2000
Long: Multiple lines
  with leading
 space
Short1: Single line value
TestBin: base64
 eJw=
 c3RyxIAMcBAFAG55BXk=
\end{verbatim}

\section{Standard keys}

In order to provide a basic level of interoperability between all systems using
the Apport report format, a number of standard key names and semantics are
defined. This is particularly important for tools which automatically reprocess
problem reports.

Implementations can add additional fields at will, especially if these are
mainly aimed at human examination. Field names which will be processed
mechanically should be added to this standard document eventually.

\subsection{Generic fields}

The following keys apply to all types of problem reports. They classify the
problem type and give information about the date, operating system and user
environment.

\begin{description}
    \item [ProblemType:] (required) Classification of the problem type;
    Currently defined values are \verb!Crash!, \verb!Kernel!, \verb!Package!
    (for failed install/upgrade of a software package), and \verb!Bug! (for
    general bug reports)

    \item [Date:] (required) Date and time of the problem report in ISO format
    (see \verb!asctime(3)!)

    \item [Uname:] (required) Output of \verb!uname -srm!

    \item [OS:] (optional) Name of the operating system. On LSB compliant
    systems, this can be determined with \verb!lsb_release -si!.

    \item [OSRelease:] (optional) Release version of the operating system. On
    LSB compliant systems, this can be determined with \verb!lsb_release -sr!.

    \item [Architecture:] (optional) OS specific notation of
    processor/system architecture (e. g. \verb!i386!)

    \item [UserGroups:] (optional) System groups of the user reporting the
    problem; for privacy reasons this should only include IDs smaller than 500,
    no groups which belong to other real users.
\end{description}

\subsection{Process specific data fields}

The following fields describe interesting properties of a particular process.
This always applies to \verb!ProblemType!s \verb!Crash! and also to \verb!Bug!
if the bug is reported against a running process (as opposed to just a
package).

\begin{description}
    \item [ExecutablePath:] (required) Contents of \verb!/proc/pid/exe! for ELF
    files; if the process is an interpreted script, this is the script path instead

    \item [InterpreterPath:] (required for scripts) Contents of
    \verb!/proc/pid/exe! if the process is an interpreted script

    \item [ProcEnviron:] (required) A subset of the process' environment, from
    \verb!/proc/pid/env!; this should only show some standard variables that do
    not disclose potentially sensitive information, like \verb!$SHELL!,
    \verb!$PATH!, \verb!$LANG!, and \verb!$LC_*!.

    \item [ProcCmdline:] (required) Contents of \verb!/proc/pid/cmdline!

    \item [ProcStatus:] (required) Contents of \verb!/proc/pid/status!

    \item [ProcMaps:] (required) Contents of \verb!/proc/pid/maps!
\end{description}

\subsection{Signal crash specific data fields}

The following fields describe properties of a process that crashed due to a
signal. This applies to \verb!ProblemType! \verb!Crash! if a core dump is
available. Note that \verb!Crash! is also used for unhandled exceptions of
programs written in scripting languages, in which case there is no core dump.

\begin{description}
    \item [CoreDump:] (optional) core dump (binary value); this can also be a
    'minidump' format or any other useful image of the stack.

    \item [Stacktrace:] (optional) Stack trace (e. g. produced by gdb's
    \verb!bt full! command or minidump processor)

    \item [ThreadStacktrace:] (optional) Threaded stack trace (e. g. produced
    by the gdb command \verb!thread apply all bt full! or minidump processor)

    \item [StacktraceTop:] (optional) First five frames of \verb!Stacktrace!
    with the leading addresses and local variables removed; this is intended to
    be evaluated for automatic duplicate detection 

    \item [Registers:] (optional) Register dump (e. g. produced by gdb's 
    \verb!info registers! command)

    \item [Disassembly:] (optional) Disassembly of the code leading to the
    crash (e. g. produced by gdb's \verb!x/16i $pc! command)
\end{description}

Note that every crash report must contain \verb!CoreDump! or a symbolic
\verb!Stacktrace! in order to be useful at all. The recommended approach is to
include the stack trace for the initial report, and drop it once it has been
recombined with debug symbols to produce a full Stacktrace.

\subsection{Package specific data fields}

The following fields describe properties of a package and its dependencies.
This applies to \verb!ProblemType!s \verb!Crash!, \verb!Package!, and
\verb!Bug! if the bug applies to a particular package (as opposed to being a
generic OS bug).

\begin{description}
    \item [Package:] (required) Package name and version, separated by space

    \item [PackageArchitecture:] (required) Processor architecture the package
    was built for; there are some architectures (like \verb!x86_64! or
    \verb!sparc64!) which support multiple package architectures

    \item [Dependencies:] (required) Package names and versions of all
    transitive dependencies of the package; one line per package

    \item [SourcePackage:] (optional) The name of the corresponding source package
\end{description}

Optionally, the name and version in \verb!Package! and \verb!Dependencies! can
be followed by a list of modified files in that package, enclosed in brackets.
Example:

\begin{verbatim}
Package: bash 3.2-1
Dependencies: libreadline5 5.2-3 [modified: /lib/libreadline.so.5]
 libc6 2.5-1 [modified: /etc/ld.so.conf]
\end{verbatim}

\subsection{Kernel specific data fields}

The following fields describe properties of a kernel oops/crash.
This applies to \verb!ProblemType! \verb!Kernel!.

\begin{description}
    \item [ProcVersion:] (required) Contents of \verb!/proc/version!
    \item [ProcCpuinfo:] (required) Contents of \verb!/proc/cpuinfo!
    \item [ProcModules:] (required) Contents of \verb!/proc/modules!
    \item [ProcCmdline:] (required) Contents of \verb!/proc/cmdline!
    \item [Dmesg:] (required) Output of \verb!dmesg!
    \item [LsPciVV:] (optional) Output of \verb!lspci -vv!
    \item [LsPciVVN:] (optional) Output of \verb!lspci -vvn!
\end{description}

\end{document}
